\documentclass[11pt, spanish]{amsart}

\usepackage[margin=2cm]{geometry}
\usepackage[utf8]{inputenc}
\usepackage{babel}
\usepackage{graphicx}
\usepackage{amsfonts}
\usepackage{amssymb}
\usepackage{amsmath}
\usepackage{amsthm}
\usepackage{mathtools}
\usepackage{xargs}
\usepackage{graphicx} % Required for inserting images
\usepackage{enumerate}
\usepackage[
    unicode=true,
    pdfusetitle,
    bookmarks=true,
    bookmarksnumbered=false,bookmarksopen=false,
    breaklinks=false,
    pdfborder={0 0 0},
    pdfborderstyle={},
    backref=section,
    colorlinks=true]{hyperref}
\newtheorem{thm}{Teorema}
\newtheorem{rem}{Observación}
\newtheorem{cor}{Corolario}
\newtheorem{prop}{Proposición}
\newtheorem{assumption}{Observación}

\newcommand{\R}{\mathbb{R}}
\numberwithin{equation}{section}
\numberwithin{figure}{section}

\begin{document}

\title{Problema de optimización del costo descontado. Estabilidad con respecto
a métricas débiles}
\begin{abstract}
Encontramos desigualdades para estimar la estabilidad (robustez) de
un problema de optimización del costo descontado para procesos de
control de Markov a tiempo discreto sobre un espacio de etapas de
Borel. El costo de una etapa se permite ser no acotado. A diferencia
de los resultados conocidos en esta área consideraremos una pertubación
de las probabilidades de transición medidas por la métrica Kantorovich,
cercanamente relacionado con la convergencia débil. Los resultados
obtenidos hacen posible estimar la tasa de desaparición del indice
de estabilidad cuando la aproximación se hace a través de medidas
empiricas. 
\end{abstract}

\maketitle
%

\section{Motivación y establecimiento del problema}

Consideremos un problema de control estándar para un proceso de Markov
a tiempo discreto (Veasé, por ejemplo, Dynkin and Yushkevich 1997;
Hernández-Lerma and Lasserre 1999). El costo total descontado es usado
como criterio de optimización. 

Suponga, además, que el proceso de control esta dado por las ecuaciones
\begin{equation}
x_{t}=F\left(x_{t-1},a_{t},\xi_{t}\right),t=1,2,\ldots,\label{eq:control_process}
\end{equation}
donde $F$ es una función medible dada. 

\[
x_{t-1},x_{t}\in X,a_{t}\in A\left(x_{t-1}\right)\subset A,t\geq1,
\]

y $\xi_{1},\xi_{2},\ldots$ es una sucesión de vectores aleatorios
i.i.d que toman valores en el espacio de Borel $S$. Denotamos $\xi$
como el vector aleatorio genérico para $\xi_{1},\xi_{2},\ldots$ El
espacio de etapas $X$ y el espacio de acciones $A$ son espacios
de Borel, y también para cada estado $x\in X$ el conjunto de acciones
admisible $A\left(x\right)$ se asume compacto, mientras $\mathbb{K}\coloneqq\left\{ \left(x,a\right):x\in X,a\in A\left(x\right)\right\} $
es medible en $X\times A$. 

El costo por etapa $c\left(x,a\right),\left(x,a\right)\in\mathbb{K}$
se asume medible. Sea $x\in X$ un estado inicial dado del proceso
y $\pi$ una politica de control seleccionada (Vea, Dynkin and Yushkevich
1997; Hernández-Lerma and Lasserre 1999 para las definiciones). Sea
$\alpha\in\left(0,1\right)$ un factor de descuento y $E_{x}^{\pi}$
denota el operador esperanza correspondiente a una probabilidad sobre
las trayectorias del proceso cuando usamos la política $\pi$ con
el estado inicial $x$. El costo total descontado es definido como
siempre:

\begin{equation}
V\left(x,\pi\right)\coloneqq E_{x}^{\pi}\sum_{i=1}^{\infty}\alpha^{t-1}c\left(x_{t-1},a_{t}\right).\label{eq:total_discounted_cost}
\end{equation}

Luego vamos a imponer condiciones para garantizar la finitud del valor
de la función:

\begin{equation}
V_{*}\left(x\right)\coloneqq\inf_{\pi\in\sqcap}V\left(x,\pi\right)\ x\in X,\label{eq:value_function}
\end{equation}

y la existencia de una política óptima $\pi_{*}$ tal que 

\begin{equation}
V\left(x,\pi_{*}\right)=V_{*}\left(x\right)\text{ for all }x\in X.\label{eq:optimal_policy}
\end{equation}

Aquí y luego $\sqcap$ denota la clase de todas las politicas de control. 

Para declarar el problema de estabilidad (o robustez) estamos interesafdos
en, asumamos que la distribución $D_{\xi}$ del vector aleatorio $\xi$
es desconocido por el controlador, pero él/ella tiene alguna aproximación
$D_{\widetilde{\xi}}$ para $D_{\xi}$ disponible. Aquí $\widetilde{\xi}$
es un vector aleatorio genérico para una sucesión de vectores aleatorios
$\widetilde{\xi}_{1},\widetilde{\xi}_{2},\ldots$ i.i.d con distribución
común $D_{\widetilde{\xi}}$. Esta situación es común en las aplicaciones. 

Todo esto significa que en lugar del proceso $x_{t}$ en (\ref{eq:control_process})
el controlador trata con la aproximación $\widetilde{x}_{t}$ dada
por la siguiente ecuación:

\begin{equation}
\widetilde{x_{t}}=F\left(\widetilde{x}_{t-1},\widetilde{a}_{t},\widetilde{\xi}_{t}\right),\ t=1,2,\ldots\label{eq:aprox_control_process}
\end{equation}

Asuamos que la única diferencia entre los dos modelos de control mostrados
arriba (originados en (\ref{eq:control_process}) y en (\ref{eq:aprox_control_process}))
radica en la diferente distribuciones de $\xi$ y $\widetilde{\xi}$,
donde lo último esta a mano. De este modo en lugar de la inaccesible
política óptima $\pi_{*}$ (satisfaciendo (\ref{eq:optimal_policy}))
uno puede intentar usar la polítca óptima $\widetilde{\pi}_{*}$ para
(\ref{eq:aprox_control_process}) como una aproximación para $\pi_{*}$.
La política $\widetilde{\pi}_{*}$ (probando si existe) satisface
las siguientes igualdades. 
\begin{equation}
\widetilde{V}\left(x,\widetilde{\pi}_{*}\right)=\widetilde{V}_{*}\left(x\right)\coloneqq\inf_{\pi\in\sqcap}\widetilde{V}\left(x,\pi\right),\text{ }x\in X,\label{eq:tilde_value_function}
\end{equation}
donde 
\begin{equation}
\widetilde{V}\left(x,\pi\right)\coloneqq E_{x}^{\pi}\sum_{t=1}^{\infty}\alpha^{t-1}c\left(\widetilde{x}_{t-1},\widetilde{a}_{t}\right).\label{eq:tilde_discounted_cost}
\end{equation}

La calidad de tal aproximación es medida por el costo adicional en
exceso de $V_{*}$ en (\ref{eq:value_function}) es decir, por la
cantidad 
\begin{equation}
\Delta\left(x\right)\coloneqq V\left(x,\widetilde{\pi}_{*}\right)-V\left(x,\pi\right)\geq0,\label{eq:stability_index}
\end{equation}
esto lo llamaremos indice de estabilidad (vea Gordienko y Salem 1998,
Gordienko y Yushkevich 2003).

Como en Gordienko y Salem (1998), Gordienko y Yushkevich 2003 el problema
de estimación de la estabilidad se declara como la búsqueda por límites
del siguiente tipo para cual nosnotros nos referiremos como desigualdades
de estabilidad 
\begin{equation}
\Delta\left(x\right)\leq C\left(x\right)\psi\left[\mu\left(D_{\xi},D_{\widetilde{\xi}}\right)\right],\text{ }x\in X,\label{eq:stability_inequalities}
\end{equation}
donde $\psi\left(s\right)\to0$ como $s\to0$, y $\mu$ es una métrica
en el espacio de medidas de probabilidad. A pesar de que $D_{\xi}$
se supone desconocida, limites superiores para $\mu\left(D_{\xi},D_{\widetilde{\xi}}\right)$
a veces pueden ser encontradas. Esto es, por ejemplo, el caso cuando
$D_{\widetilde{\xi}}=\widehat{D}\left(\xi_{1},\xi_{2},\ldots,\xi_{n}\right)$
es la distribución empírica y $\mu$ es una adecuada métrica ``débil''
(es decir, relacioada con la convergencia débil de medidas).

Los resultados en los articulos (Gordienko y Salem 1998,2000; Gordienko
y Yushkevich 2003; Montes-de-Oca y Salem-Silva 2005; Montes-de-Oca
et al. 2003) proporcionan desigualdades como en (\ref{eq:stability_inequalities})
para $\psi\left(s\right)=s^{\gamma}\ \left(0<\gamma\leq1\right)$
con las llamdas ``métricas fuertes'' tales como
\begin{itemize}
\item La métrica de la variación total.
\[
\mu\left(D_{\xi},D_{\widetilde{\xi}}\right)\coloneqq\sup\left\{ \left|E\left[\varphi\left(\xi\right)-\varphi\left(\widetilde{\xi}\right)\right]\right|:\|\varphi\|_{\infty}\leq1\right\} 
\]
\item La métrica de la variación total w-ponderada.
\[
\mu_{w}\left(D_{\xi},D_{\widetilde{\xi}}\right)\coloneqq\sup\left\{ \left|E\left[\varphi\left(\xi\right)-\varphi\left(\widetilde{\xi}\right)\right]\right|:\sup_{s\in S}\dfrac{\left|\varphi\left(s\right)\right|}{w\left(s\right)}\leq1\right\} 
\]
\end{itemize}
%
Para resultados relacionados, vea también (Van Dijk 1988; Van Dijk
and Sladky 1999). Desafortunadamente los resutados mencionados son
inutiles, por ejemplo, cuando usamos la métrica empírica $\widehat{D}\left(\xi_{1},\xi_{2},\ldots,\xi_{n}\right)$
como la aproximación conocida de $D_{\xi}$. El objetivo de esté artículo
es presentar un conjunto de condiciones sobre los modelos de control
bajo consideración que permita probar (\ref{eq:stability_inequalities})
para $\psi\left(s\right)=s$ y cierta metríca de probabilidad débil
$\mu$. Por ejemplo, El teorema (\ref{thm:Teo2}) en la siguiente
sección asegura que bajo condiciones apropiadas 
\begin{equation}
\Delta\left(x\right)\leq K\left(x\right)\ell\left(D_{\xi},D_{\widetilde{\xi}}\right),\ x\in X,\label{eq:apropiate_stab_inequality}
\end{equation}
donde 
\begin{equation}
\ell\left(D_{\xi},D_{\widetilde{\xi}}\right)\coloneqq\sup\left\{ \left|E\left[\varphi\left(\xi\right)-\varphi\left(\widetilde{\xi}\right)\right]\right|:\varphi\text{ tal que }\left|\varphi\left(s\right)-\varphi\left(s'\right)\right|\leq r\left(s,s'\right),\ s,s'\in S\right\} ,\label{eq:Kantorovich_metric}
\end{equation}
y $r$ es la métrica en el espacio $S$. 

Esta distancia es conocida como la métrica de Kantorovich. Una sucesión
$\ell-$convergente de vectores aleatorios converge débilmente. Esto
es bien conocido (vea, por ejemplo Rachev and Rüschendorf 1998) en
el caso $S=\mathbb{R}^{k}$ que $\ell\left(D_{\xi},D_{\widetilde{\xi}}\right)\to0$
si y solo si $\xi_{n}\Rightarrow\xi$ y $E\left|\xi_{n}\right|\to E\left|\xi\right|$.

Cuando la métrica empírica $\widehat{D}\left(\xi_{1},\xi_{2},\ldots,\xi_{n}\right)$
son usadas para aproximar la distribución $D_{\xi}$ sobre $\left(\mathbb{R}^{k},\mathcal{B}\left(\mathbb{R}^{k}\right)\right)$
vamos a probar que 
\begin{equation}
E\Delta\left(x\right)\leq\widetilde{K}\left(x\right)n^{-\delta\left(k\right)}\ln n,\ n=2,3,\ldots,x\in X\label{eq:mean_stab_inequality}
\end{equation}

proporcionando que $Ee^{\gamma\left|\xi\right|}<\infty$ para álgun
$\gamma>0$. En (\ref{eq:mean_stab_inequality}) el exponente $\delta\left(k\right)$
se comporta como $1/k$. vale la pena mencionar que existen ejemplos
de procesos de control de Markov (vea Gordienko y Salem 1998) con
costo total descontado inestable finito con respecto a este criterio.
Para tales procesos desigualdades como (\ref{eq:apropiate_stab_inequality})
y (\ref{eq:mean_stab_inequality}) son imposibles.

En la tercera sección de esté artículo daremos dos ejemplos de procesos
de control de Markov para cuales las estimaciones (\ref{eq:apropiate_stab_inequality})
y (\ref{eq:mean_stab_inequality}) son válidas. Uno de ellos es un
modelo de inventario o un modelo de cola con tasa de servicio controlada
por el sistema. El otro es el modelo de optimización de cartera en
tiempo discreto. (Korn and Korn 2001)

\section{Suposiciones y resultados\label{sec:Suposiciones-y-resultados}}

Recordemos, $X$ como el espacio de etapas, el espacio de acciones
$A$ y el espacio de ruido $S$ (El espacio donde las perturbaciones
aleatorias $\xi_{n}$ toman valores) son todos los subconjuntos de
sus espacios métricos respectivos, ellos mismos son espacios métricos
con sus correspondientes métricas: $\left(X,\rho\right),\left(A,d\right),\left(S,r\right)$. 

La distancia de Hausdorff entre subconjuntos compactos $B,C$ de $A$
esta dado por la expresión 

\[
h\left(B,C\right)\coloneqq\max_{x\in B}\left\{ \sup_{x\in B}d\left(x,C\right),\sup_{y\in C}d\left(y,B\right)\right\} .
\]

El primer conjunto de suposiciones técnicas es prestada de Hernández-Lerna
y Lasserre $\left(1999\right)$ y es necesarian para asegurar la existencia
de minimizadores en las ecuaciones de optimalidad correspondientes.
\begin{assumption}
.\label{assu:Sup1}

1. El conjutno $A\left(x\right)$ es compacto para cada $x\in X$
y el mapeo de valores establecidos $x\mapsto A\left(x\right)$ es
semicontinua superior con respecto a la métrica de Haussdorf.

2. La función de costo $c:\mathbb{K}\to\mathbb{R}$ es semicontinua
inferior. 

3. Para cada función continua acotada $u:X\to\mathbb{R}$ las funciones
\[
u'\left(x,a\right)\coloneqq Eu\left[F\left(x,a,\xi\right)\right],
\]
y 
\[
u''\left(x,a\right)\coloneqq Eu\left[F\left(x,a,\xi\right)\right]
\]
son continuas en $\mathbb{K}$. 
\end{assumption}
%
El segundo conjunto de condiciones es usado para proporcionar la existencia
de las soluciones para la ecuación de optimalidad y para asegurar
algunas propiedades útiles de esa solución. (Vea Gordienko and Salem
1998; Hernández Lerma and Lasserre 1999 y la demostración del Teorema
$1$ en la sección $4$)
\begin{assumption}
\label{assu:Sup2}Existe una función $W:X\to[1,\infty)$ tal que

1. 
\begin{equation}
\left|c\left(x,a\right)\right|\leq W\left(x\right),\text{ para cada }\left(x,a\right)\in\mathbb{K}\label{eq:c_w_bounded}
\end{equation}

2. Exise una constante $\beta\in\left(\alpha,1\right)$ tal que 
\begin{align}
EW\left[F\left(x,a,\xi\right)\right] & \leq\dfrac{\beta}{\alpha}W\left(x\right),\label{eq:W_xi}\\
EW\left[F\left(x,a,\widetilde{\xi}\right)\right] & \leq\dfrac{\beta}{\alpha}W\left(x\right),\label{eq:w_sxi}
\end{align}
para cada $\left(x,a\right)\in\mathbb{K}$. 

3. Las funciones 

\[
w'\left(x,a\right)\coloneqq EW\left[F\left(x,a,\xi\right)\right]
\]
y 
\[
w''\left(x,a\right)\coloneqq EW\left[F\left(x,a,\widetilde{\xi}\right)\right]
\]

son continuas en $\mathbb{K}$. 
\end{assumption}
%
\begin{rem}
Es conocido (vea Hernández-Lerna y Lasserre 1999; Van Nunen and Wessels
1978) que las siguientes condiciones sonn suficientes para (\ref{eq:c_w_bounded})-(\ref{eq:w_sxi}).

Existe uan función continua $W_{1}:X\to[1,\infty)$ y constantes $\gamma,\overline{c}$
y $b$ tales que 
\end{rem}
\begin{itemize}
\item $\alpha\gamma<1$
\item .
\begin{equation}
\left|c\left(x,a\right)\right|\leq\overline{c}W_{1}\left(x\right),\ (x,a)\in\mathbb{K}\label{eq:w_1_c_bounded}
\end{equation}
\item .
\begin{equation}
EW_{1}\left[F\left(x,a,\xi\right)\right]\leq\gamma W_{1}\left(x\right)+b\label{eq:Lyapu1}
\end{equation}
\item .
\begin{equation}
EW_{1}\left[F\left(x,a,\widetilde{\xi}\right)\right]\leq\gamma W_{1}\left(x\right)+b.\ \left(x,a\right)\in\mathbb{K}\label{eq:Lyapu2}
\end{equation}
\end{itemize}
%
Denotaremos por $B_{w}$ el espacio de Banach de todas las funciones
medibles $u:X\to\mathbb{R}$ para las cuales 
\[
\|u\|_{w}=\sup_{x\in X}\dfrac{\left|u\left(x\right)\right|}{W\left(x\right)}
\]
 es finita. 

De Hernández-Lerma y Lasserre (1999), p. 47, Teorema 8.3.6, sigue
que bajo las suposiciones (\ref{assu:Sup1}) y (\ref{assu:Sup2})
para ambos procesos de control (\ref{eq:control_process}) y (\ref{eq:aprox_control_process})
existe su política estacionaria óptima respectiva. 
\begin{align*}
\pi_{*} & =\left\{ f_{*},f_{*},\ldots\right\} ,a_{t}=f_{*}\left(x_{t-1}\right),t\geq1;\\
\widetilde{\pi}_{*} & =\left\{ \widetilde{f}_{*},\widetilde{f}_{*},\ldots\right\} ,a_{t}=\widetilde{f}_{*}(\widetilde{x}_{t-1}),t\ge1;
\end{align*}
con su valores de función $V_{*}\left(x\right)=V\left(x,\pi_{*}\right)\in B_{w}$
y $\widetilde{V_{*}}\left(x\right)=\tilde{V}\left(x,\widetilde{\pi}_{*}\right)\in B_{w}$
semicontinua inferior correspondiente, y sus esperanzas $EV_{*}\left[F\left(x,a,\xi\right)\right],EV_{*}\left[F\left(x,a,\widetilde{\xi}\right)\right]$
existe para toda $\left(x,a\right)\in\mathbb{K}$. Denotamos por $M$
al conjunto de todas las distribuciones de $\xi$ y $\widetilde{\xi}$
para la cual las esperanzas de arriba existen para toda $\left(x,a\right)\in\mathbb{K}$.
Definiremos la siguiente pseudo-métrica $\mu$ en $M$. 
\begin{equation}
\mu\left(D_{\xi},D_{\widetilde{\xi}}\right)\coloneqq\sup\left\{ \left|EV_{*}\left[F\left(x,a,\xi\right)\right]-EV_{*}\left[F\left(x,a,\widetilde{\xi}\right)\right]\right|:\left(x,a\right)\in\mathbb{K}\right\} \label{eq:pseudo-metric}
\end{equation}

Esta pseudo-métrica toma valores en $[0,\infty]$. También $\mu\left(D_{\xi},D_{\widetilde{\xi}}\right)$
podría ser cero para distribuciones no idénticas $D_{\xi},D_{\widetilde{\xi}}$.
\begin{thm}
\label{thm:Teo1} Bajo las suposiciones (\ref{assu:Sup1}) y (\ref{assu:Sup2})
tenemos: 
\begin{equation}
\Delta\left(x\right)\leq2\alpha\left[\left(1-\alpha\right)^{-1}+\alpha\left(1-\beta\right)^{-2}W\left(x\right)\right]\mu\left(D_{\xi},D_{\widetilde{\xi}}\right),x\in X.\label{eq:Teo1}
\end{equation}
\end{thm}
%
\begin{proof}
Simplificando la notación de sección (\ref{sec:Suposiciones-y-resultados})
sea $\pi=\left\{ f,f,\ldots\right\} ,\widetilde{\pi}=\left\{ \widetilde{f},\widetilde{f},\ldots\right\} $
las políticas estacionarias óptimas para los procesos (\ref{eq:control_process})
y (\ref{eq:aprox_control_process}), respectivamente, y $V_{*},\widetilde{V}_{*}$
sean las funciones de valor correspondientes. (Vea (\ref{eq:value_function})
y (\ref{eq:tilde_value_function})). 

Entonces (vea Hernández-Lerma y Lasserre 1999, Cap. 8) $V_{*},\widetilde{V}_{*}$
y $f,\widetilde{f}$ satisfacen las siguientes ecuaciones de optimalidad.
(Además, $V_{*}$ y $\widetilde{V}_{*}$ son las únicas soluciones
para estas ecuaciones)

\begin{align}
V_{*}\left(x\right) & =\inf_{a\in A\left(x\right)}\left\{ c\left(x,a\right)+\alpha EV_{*}\left[F\left(x,a,\xi\right)\right]\right\} \nonumber \\
 & =\inf_{a\in A\left(x\right)}\left\{ c\left(x,f\left(x\right)\right)+\alpha EV_{*}\left[F\left(x,f\left(x\right),\xi\right)\right]\right\} \label{eq:proof1vstar}\\
\tilde{V}_{*}\left(x\right) & =\inf_{a\in A\left(x\right)}\left\{ c\left(x,a\right)+\alpha E\widetilde{V}_{*}\left[F\left(x,a,\widetilde{\xi}\right)\right]\right\} \nonumber \\
 & =\inf_{a\in A\left(x\right)}\left\{ c\left(x,\widetilde{f}\left(x\right)\right)+\alpha E\widetilde{V}_{*}\left[F\left(x,\widetilde{f}\left(x\right),\widetilde{\xi}\right)\right]\right\} \label{eq:proof1tilde_vstar}
\end{align}

Para $\left(x,a\right)\in\mathbb{K}$ definimos 
\begin{align}
H\left(x,a\right) & \coloneqq c\left(x,a\right)+\alpha EV_{*}\left[F\left(x,a,\xi\right)\right]\nonumber \\
\widetilde{H}\left(x,a\right) & \coloneqq c\left(x,a\right)+\alpha E\widetilde{V}_{*}\left[F\left(x,a,\widetilde{\xi}\right)\right]\label{eq:Hxa}
\end{align}

y sea $\Gamma_{t}=\left\{ x,a_{1},x_{1},a_{2},\ldots,x_{t-1},a_{t}\right\} ,\left(t\geq1\right)$
la parte de una trayectoria del proceso (\ref{eq:control_process})
bajo la política de control $\widetilde{\pi}=\left\{ \widetilde{f},\widetilde{f},\ldots\right\} $,
por la propiedad de Markov del proceso. 
\begin{align*}
\zeta_{t} & \coloneqq E^{\widetilde{\pi}}\left[\alpha V_{*}\left(x_{t}\right)\mid\Gamma_{t}\right]\\
 & =H\left(x_{t-1},a_{t}\right)-c\left(x_{t-1},a_{t}\right)-\inf_{a\in A\left(x_{t-1}\right)}H\left(x_{t-1},a\right)+\inf_{a\in A\left(x_{t-1}\right)}H\left(x_{t-1},a\right).
\end{align*}
En vista de (\ref{eq:proof1vstar}) y (\ref{eq:Hxa}) obtenemos: 
\begin{align*}
\zeta_{t} & =H\left(x_{t-1},a_{t}\right)-\inf_{a\in A\left(x_{t-1}\right)}H\left(x_{t-1},a\right)-c\left(x_{t-1},a_{t}\right)+V_{*}\left(x_{t-1}\right)\\
 & =\Lambda_{t}-c\left(x_{t-1},a_{t}\right)+V_{*}\left(x_{t-1}\right),
\end{align*}
donde 
\begin{equation}
\Lambda_{t}\coloneqq H\left(x_{t-1},a_{t}\right)-\inf_{a\in A\left(x_{t-1}\right)}H\left(x_{t-1},a\right).\label{eq:Lambda_t}
\end{equation}

De este modo 
\begin{align}
E_{x}^{\widetilde{\pi}}\alpha V_{*}\left(x_{t}\right) & =E_{x}^{\widetilde{\pi}}\zeta_{t}\nonumber \\
 & =E_{x}^{\widetilde{\pi}}\Lambda_{t}-E_{x}^{\widetilde{\pi}}c\left(x_{t-1},a_{t}\right)+E_{x}^{\widetilde{\pi}}V_{*}\left(x_{t-1}\right)\label{eq:mean_lambda}
\end{align}

Sumando (\ref{eq:mean_lambda}) con los pesos $\alpha^{t-1}$. 
\[
\sum_{i=1}^{n}\alpha^{t-1}E_{x}^{\widetilde{\pi}}c\left(x_{t-1},a_{t}\right)=\sum_{i=1}^{n}\alpha^{t-1}\left[E_{x}^{\widetilde{\pi}}V_{*}\left(x_{t-1}\right)-E_{x}^{\widetilde{\pi}}\alpha V_{*}\left(x_{t}\right)\right]+\sum_{i=1}^{n}\alpha^{t-1}E_{x}^{\widetilde{\pi}}\Lambda_{t},
\]
 o 
\begin{equation}
\sum_{i=1}^{n}\alpha^{t-1}E_{x}^{\widetilde{\pi}}c\left(x_{t-1},a_{t}\right)-V_{*}\left(x\right)=-\alpha^{n}E_{x}^{\widetilde{\pi}}V_{*}\left(x_{t-1}\right)+\sum_{i=1}^{n}\alpha^{t-1}E_{x}^{\widetilde{\pi}}\Lambda_{t}.\label{eq:sum_or_sum}
\end{equation}

Desde $V^{*}\in B_{w}$ de la suposición (\ref{assu:Sup2}) se sigue
(vea Hernández-Lerma y Lasserre 1999, p. 52) que ${\displaystyle \lim_{n\to\infty}\alpha^{n}E_{x}^{\widetilde{\pi}}V_{*}\left(x_{n}\right)=0}$

Por lo tanto, pasando el límite con $n\to\infty$ en (\ref{eq:sum_or_sum})
obtenemos (vea (\ref{eq:stability_index}))

\begin{equation}
\Delta\left(x\right)=\sum_{i=1}^{\infty}\alpha^{t-1}E_{x}^{\widetilde{\pi}}c\left(x_{t-1},a_{t}\right)-V_{*}\left(x\right)=\sum_{i=1}^{\infty}\alpha^{t-1}E_{x}^{\widetilde{\pi}}\Lambda_{t}\label{eq:Delta_infinite}
\end{equation}

Por la definición de $\Lambda_{t}$ en (\ref{eq:Lambda_t}) y por
(\ref{eq:proof1tilde_vstar}) obtenemos lo siguiente 
\[
\Lambda_{t}=H\left(x_{t-1},a_{t}\right)-\widetilde{H}\left(x_{t-1},a_{t}\right)+\inf_{a\in A\left(x_{t-1}\right)}\widetilde{H}\left(x_{t-1},a\right)-\inf_{a\in A\left(x_{t-1}\right)}H\left(x_{t-1},a\right),
\]

De este modo, 
\begin{align}
\left|\Lambda_{t}\right| & \leq2\sup_{a\in A\left(x_{t-1}\right)}\left|H\left(x_{t-1},a\right)-\widetilde{H}\left(x_{t-1},a\right)\right|\nonumber \\
 & \leq2\alpha\sup_{a\in A\left(x_{t-1}\right)}\left|EV_{*}\left[F\left(x_{t-1},a,\xi\right)\right]-E\widetilde{V}_{*}\left[F\left(x_{t-1},a,\widetilde{\xi}\right)\right]\right|,\label{eq:Lambda_inequality}
\end{align}

donde la esperanza del último término es tomada con respecto a los
vectores aleatorios $\xi,\widetilde{\xi}$ con $x_{t-1}$ fijo. De
la última desigualdad obtenemos 
\begin{align}
\left|\Lambda_{t}\right| & \leq2\alpha\sup_{a\in A\left(x_{t-1}\right)}\left|EV_{*}\left[F\left(x_{t-1},a,\xi\right)\right]-EV_{*}\left[F\left(x_{t-1},a,\widetilde{\xi}\right)\right]\right|\label{eq:lamba_trick}\\
 & +2\alpha\sup_{a\in A\left(x_{t-1}\right)}\left|EV_{*}\left[F\left(x_{t-1},a,\widetilde{\xi}\right)\right]-E\widetilde{V}_{*}\left[F\left(x_{t-1},a,\widetilde{\xi}\right)\right]\right|\nonumber 
\end{align}

En vista de (\ref{eq:pseudo-metric}) el primer sumando del lado derecho
de (\ref{eq:lamba_trick}) esta acotado por $2\alpha\mu$, donde denotamos
$\mu\coloneqq\mu\left(D_{\xi},D_{\widetilde{\xi}}\right)$. El segundo
sumando es menor que 
\begin{align}
 & 2\alpha\sup_{a\in A\left(x_{t-1}\right)}E\left[\dfrac{W\left[F\left(x_{t-1},a,\widetilde{\xi}\right)\right]}{W\left[F\left(x_{t-1},a,\widetilde{\xi}\right)\right]}\left|V_{*}\left[F\left(x_{t-1},a,\widetilde{\xi}\right)\right]-\widetilde{V}_{*}\left[F\left(x_{t-1},a,\widetilde{\xi}\right)\right]\right|\right]\label{eq:trick_w}\\
 & \leq2\alpha\|V_{*}-\widetilde{V}_{*}\|_{w}\sup_{a\in A\left(x_{t-1}\right)}EW\left[F\left(x_{t-1},a,\widetilde{\xi}\right)\right]\nonumber 
\end{align}

Necesitamos acotar el promedio del último factor de (\ref{eq:trick_w}).
Usando la notación $E_{\widetilde{\xi}}$ en lugar de $E$ y la desigualdad
(\ref{eq:w_sxi}) obtenemos para cada $x_{t-1}$ 
\begin{align*}
EW\left[F\left(x_{t-1},a,\widetilde{\xi}\right)\right] & \equiv E_{\widetilde{\xi}}W\left[F\left(x_{t-1},a,\widetilde{\xi}\right)\right]\leq\dfrac{\beta}{\alpha}E_{x}^{\widetilde{\pi}}W\left(x_{t-1}\right)\\
 & =\dfrac{\beta}{\alpha}E_{x}^{\widetilde{\pi}}W\left[F\left(x_{t-2},a_{t-1},\xi_{t-1}\right)\right]\\
 & =\dfrac{\beta}{\alpha}E_{x}^{\widetilde{\pi}}\left\{ E_{x}^{\widetilde{\pi}}W\left[F\left(x_{t-2},a_{t-1},\xi_{t-1}\right)\right]\mid\Gamma_{t-1}\right\} \\
 & \leq\left(\dfrac{\beta}{\alpha}\right)^{2}E_{x}^{\widetilde{\pi}}\left\{ W\left(x_{t-2}\right)\mid\Gamma_{t-1}\right\} \\
 & =\left(\dfrac{\beta}{\alpha}\right)^{2}E_{x}^{\widetilde{\pi}}\left\{ W\left(x_{t-3},a_{t-2},\xi_{t-2}\right)\mid\Gamma_{t-1}\right\} \\
 & =\left(\dfrac{\beta}{\alpha}\right)^{2}E_{x}^{\widetilde{\pi}}W\left(x_{t-3},a_{t-2},\xi_{t-2}\right).
\end{align*}

Procediendo de forma inductiva obtenemos las desigualdades 
\begin{equation}
E_{x}^{\widetilde{\pi}}\sup_{a\in A\left(x_{t-1}\right)}E_{\widetilde{\xi}}W\left[F\left(x_{t-1},a,\widetilde{\xi}\right)\right]\leq\left(\dfrac{\beta}{\alpha}\right)^{t-1}W\left(x\right),\ t=1,2,\ldots\label{eq:super_inequality}
\end{equation}

En Gordienko y Salem (1998), Hernández-Lerma and Lasserre (1999) fue
probaod que los siguientes operadores. 
\begin{align*}
Tu\left(x\right) & \coloneqq\inf_{a\in A\left(x\right)}\left\{ c\left(x,a\right)+aEu\left[F\left(x,a,\xi\right)\right]\right\} \\
\widetilde{T}u\left(x\right) & \coloneqq\inf_{a\in A\left(x\right)}\left\{ c\left(x,a\right)+aEu\left[F\left(x,a,\widetilde{\xi}\right)\right]\right\} ,
\end{align*}
son contractivas en $B_{w}$ con módulo $\beta$. Como $V_{*}$ y
$\widetilde{V}_{*}$ son puntos fijos para esos operadores obtenemos
\[
\|V_{*}-\widetilde{V}_{*}\|_{w}=\|TV_{*}-\widetilde{T}V_{*}\|_{w}+\|\widetilde{T}V_{*}-\widetilde{T}\widetilde{V}_{*}\|_{w},
\]
 esto implica que

\begin{align}
\|V_{*}-\widetilde{V}_{*}\|_{w} & \leq\dfrac{1}{1-\beta}\|TV_{*}-\widetilde{T}V_{*}\|_{w}\nonumber \\
 & \leq\dfrac{\alpha}{1-\beta}\sup_{x\in X}W^{-1}\left(x\right)\label{eq:Vnorm}\\
 & \sup_{a\in A\left(x\right)}\left|EV_{*}\left[F\left(x,a,\xi\right)\right]-EV_{*}\left[F\left(x,a,\widetilde{\xi}\right)\right]\right|\leq\dfrac{\alpha}{1-\beta}\mu\nonumber 
\end{align}

Combinando las desigualdades \ref{eq:lamba_trick}-\ref{eq:Vnorm}
obtenemos 

\[
E_{x}^{\widetilde{\pi}}\left|\Lambda_{t}\right|\leq2\alpha\left[1+\dfrac{\alpha}{1-\beta}\left(\dfrac{\beta}{\alpha}\right)^{t-1}W\left(x\right)\right]\mu
\]

y finalmente en vista de \ref{eq:Delta_infinite} recibimos la desigualdad
\ref{eq:Teo1}.
\end{proof}
%
Nuestro siguiente paso is acotar la pseudo-métrica $\mu$ por un métrica
de probabilidad débil relevante.
\begin{assumption}
\label{assu:Sup3}Existe una constante $L_{0}$ y una función medible
$L_{1}:S\to[0,\infty)$ tal que:

1. $\left|c\left(x,a\right)-c\left(y,a\right)\right|\leq L\rho\left(x,y\right)\ \text{ for all }\left(x,a\right),\left(y,a\right)\in\mathbb{K}$.

2. $\rho\left[F\left(x,a,\xi\right),F\left(y,a,\xi\right)\right]\leq L_{1}\left(\xi\right)\rho\left(x,y\right)\text{ }\text{for all }\left(x,a\right),\left(y,a\right)\in\mathbb{K}$
y $L_{1}\coloneqq EL_{1}\left(\xi\right)\leq1$.

$3.$ $A$ es compacto y $A\left(x\right)=A$ para todo $x\in X$. 
\end{assumption}
\begin{thm}
\label{thm:Teo2} Bajo las suposiciones $1-3$ obtenemos 
\begin{equation}
\Delta\left(x\right)\leq2\alpha\dfrac{L_{0}}{1-\alpha L_{1}}\left[\dfrac{1}{1-\alpha}+\dfrac{\alpha}{\left(1-\beta\right)^{2}}W\left(x\right)\right]\ell\left(D_{\xi},D_{\widetilde{\xi}}\right),\ x\in X.\label{eq:Teo2}
\end{equation}
\end{thm}
%
\begin{proof}
En vista de la definición (\ref{eq:w_sxi}) de la métrica de Kantorovich
$\ell$ y la definición (\ref{eq:pseudo-metric}) de la métrica $\mu$,
la desigualdad (\ref{thm:Teo2}) seguiría de (\ref{eq:Teo1}) si estableemos
que para cada $\left(x,a\right)\in\mathbb{K}$ la función 
\[
\varphi\left(\cdot\right)\coloneqq V_{*}\left[F\left(x,a,\cdot\right)\right]:S\to\mathbb{R},
\]
satisface la condición de Lipschitz con constante $\dfrac{L_{0}}{1+\alpha L_{1}}$
(vea (\ref{assu:Sup3}) para la definción de $L_{0}$ y $L_{1}$).

Sea $V_{0}\coloneqq0$ y para cada $n=1,2,\ldots$ 
\[
V_{n}\left(x\right)\coloneqq\inf_{a\in A}\left\{ c\left(x,a\right)+\alpha EV_{n-1}\left[F\left(x,a,\xi\right)\right]\right\} ,\ x\in X
\]

Como esta mostrado en Hernández-Lerma y Lasserre $\left(1999\right)$,
p. 47 para cada $x\in X$ $V_{n}\left(x\right)\to V_{*}\left(x\right)$
cuando $n\to\infty$. Por la suposición (\ref{assu:Sup3}), punto
1 y $3$ obtenemos que $V_{1}\in Lip\left(L_{0}\right)$ (Es decir,
satisface la condición de Lipschitz con la constante $L_{0}$). Aplicando
la suposición 3 punto 2 obtenemos 
\begin{align*}
\left|V_{2}\left(x\right)-V_{2}\left(y\right)\right| & \leq\sup_{a\in A}\left\{ \left|c\left(x,a\right)-c\left(y,a\right)\right|+\alpha\left|EV_{1}\left[F\left(x,a,\xi\right)\right]-EV_{1}\left[F\left(y,a,\xi\right)\right]\right|\right\} \\
 & \leq L_{0}\rho\left(x,y\right)+\alpha\sup_{a\in A}EL_{0}\rho\left[F\left(x,a,\xi\right),F\left(y,a,\xi\right)\right]\\
 & \leq L_{0}\left(1+\alpha L_{1}\right)\rho\left(x,y\right),\ x,y\in X
\end{align*}

Por indución establecemos que $V_{n}\in Lip\left(\dfrac{L_{0}}{1-\alpha L_{1}}\right),n=1,2,\ldots$
y de este modo 
\[
V_{*}\in Lip\left(\dfrac{L_{0}}{1-\alpha L_{1}}\right)
\]
\end{proof}
\begin{rem}
Para $S=\mathbb{R}$. 
\[
\ell\left(D_{\xi},D_{\widetilde{\xi}}\right)=\int_{-\infty}^{\infty}\left|F_{\xi}\left(x\right)-F_{\widetilde{\xi}}\left(x\right)\right|\mathrm{d}x,
\]
donde $F_{\xi},F_{\widetilde{\xi}}$ son las funciones de distribución
correspondientes (vea Rachev y Rüschendorf 1998). 

Consideremos la aproximación de $D_{\widetilde{\xi}}$ por medidas
empíricas. Sea $n\geq1$ fijo, $\xi_{1},\xi_{2},\ldots,\xi_{n}$ vectores
aleatorios i.i.d con distribución $D_{\xi}$ y 
\[
\widehat{D}_{n}=\widehat{D}\left(\xi_{1},\ldots,\xi_{n}\right)\coloneqq\dfrac{1}{n}\sum_{i=1}^{n}\delta_{\xi_{i}},
\]
sea la medida empírica (esto es una medida aleatoria en $\left(S,\mathcal{B}\left(S\right)\right)$).
Denotamos por $\widetilde{\xi}^{\left(n\right)}$ un vector aleatorio
con distribución $\widehat{D}_{n}$ y sea $\widetilde{\xi}_{t}^{\left(n\right)},t=1,2,\ldots$
copias independientes de $\widetilde{\xi}^{\left(n\right)}$.

Supongamos que $\widetilde{\xi}^{\left(n\right)}\equiv\widetilde{\xi}_{t},t=1,2,\ldots$
en el proceso de aproximación (\ref{eq:aprox_control_process}). Entonces
el indice de estabilidad (\ref{eq:stability_index}) es aleatorio.
Para evitar problemas de medibilidad la esperanza tomada en (\ref{eq:Corolario})
y (\ref{eq:Prop1}) a continuación puede tratarse como una integral
externa como se define en Van Der Vaart y Wellner (1996).
\end{rem}
%
\begin{cor}
\label{cor:Corolario1}

\begin{equation}
E\Delta\left(x\right)\leq K\left(x\right)n^{-\delta\left(\epsilon,k\right)}\ln n,\ n=2,3,\ldots,x\in X\label{eq:Corolario}
\end{equation}
\end{cor}
%
\begin{prop}
s

\begin{equation}
E\sup_{\varphi\in Lip}\left|E\varphi\left(\xi\right)-\dfrac{1}{n}\sum_{i=1}^{n}\varphi\left(\xi_{i}\right)\right|\leq Mn^{-\delta\left(\epsilon,k\right)}\ln\left(n\right),\ n=2,3,\ldots,\label{eq:Prop1}
\end{equation}
\end{prop}


\end{document}